%% Adaptado a partir de :
%%    abtex2-modelo-trabalho-academico.tex, v-1.9.2 laurocesar
%% para ser um modelo para os trabalhos no IFSP-SPO

\documentclass[
    % -- opções da classe memoir --
    12pt,               % tamanho da fonte
    openright,          % capítulos começam em pág ímpar (insere página vazia caso preciso)
    %twoside,            % para impressão em verso e anverso. Oposto a oneside
    oneside,
    a4paper,            % tamanho do papel. 
    % -- opções da classe abntex2 --schwinn
    % Opções que não devem ser utilizadas na versão final do documento
    %draft,              % para compilar mais rápido, remover na versão final
    paginasA3,  % indica que vai utilizar paginas em A3 
    MODELO,             % indica que é um documento modelo então precisa dos geradores de texto
    TODO,               % indica que deve apresentar lista de pendencias 
    % -- opções do pacote babel --
    english,            % idioma adicional para hifenização
    brazil              % o último idioma é o principal do documento
    ]{ifsp-spo-inf-ctds} % ajustar de acordo com o modelo desejado para o curso

% ---
% Pacotes básicos 
% ---
%\usepackage[utf8]{inputenc}     % Codificacao do documento (conversão automática dos acentos)
% ---

%\usepackage{style}
        

% --- 
% CONFIGURAÇÕES DE PACOTES ADICIONAIS UTEIS
% --- 


% ---
% Informações de dados para CAPA e FOLHA DE ROSTO
% ---
\titulo{ANÁLISE DOS MECANISMOS DE CONTRATAÇÃO DE FUNCIONÁRIOS DA INQUESTI}

% Trabalho individual
%\autor{AUTOR DO TRABALHO}

% Trabalho em Equipe
% ver também https://github.com/abntex/abntex2/wiki/FAQ#como-adicionar-mais-de-um-autor-ao-meu-projeto
\renewcommand{\imprimirautor}{
\begin{tabular}{lr}
DOUGLAS RODRIGUES PEREZ & SP1665626 \\
\end{tabular}
}


\disciplina{A5RHU - Recursos Humanos}

\preambulo{Trabalho da disciplina de A5RHU com o objetivo de 
realizar uma análise dos mecanismos de contratação de funcionários da empresa em que o autor trabalhe.}

\data{06/09/2020}

% Definir o que for necessário e comentar o que não for necessário
% Utilizar o Nome Completo, abntex tem orientador e coorientador
% então vão ser utilizados na definição de professor
\renewcommand{\orientadorname}{Professor:}
\orientador{LUIS FERNANDO AIRES BRANCO MENEGUETI}

% ---


% informações do PDF
\makeatletter
\hypersetup{
        %pagebackref=true,
        pdftitle={\@title}, 
        pdfauthor={\@author},
        pdfsubject={\imprimirpreambulo},
        pdfcreator={LaTeX with abnTeX2 using IFSP model},
        pdfkeywords={abnt}{latex}{abntex}{abntex2}{IFSP}{\ifspprefixo}{trabalho acadêmico}, 
        colorlinks=true,            % false: boxed links; true: colored links
        linkcolor=blue,             % color of internal links
        citecolor=blue,             % color of links to bibliography
        filecolor=magenta,              % color of file links
        urlcolor=blue,
        bookmarksdepth=4
}
\makeatother
% --- 


% ----
% Início do documento
% ----
\begin{document}

% Retira espaço extra obsoleto entre as frases.
\frenchspacing 

% -- lista de pendencias gerada pelo todonotes
% -- altere opções do usepackage para remover na versão final....
\newpage

% ----------------------------------------------------------
% ELEMENTOS PRÉ-TEXTUAIS
% ----------------------------------------------------------
\pretextual

% ---
% Capa
% ---
\imprimircapa


% ---

% ---
% Folha de rosto
% (o * indica que haverá a ficha bibliográfica)
% ---
\imprimirfolhaderosto

% ---
% Dedicatória
% ---
\begin{dedicatoria}
   \vspace*{\fill}
   \centering
   \noindent
   \textit{ Este trabalho é dedicado às crianças adultas que,\\
   quando pequenas, sonharam em se tornar cientistas.} 

\todonum{colocar sua dedicatoria}
   
   \vspace*{\fill}
   

\end{dedicatoria}
% ---
% ---
% Agradecimentos
% ---
\begin{agradecimentos}
Os agradecimentos principais são direcionados à Mylena Martins de Oliveira, meus pais e todos os meus amigos e colegas que estão ao meu lado.
\end{agradecimentos}
% ---
% ---
% Epígrafe
% ---
\begin{epigrafe}
    \vspace*{\fill}
    \begin{flushright}
        \textit{``Eu me tornei insano, \\
        com longos intervalos de horrível sanidade.\\
        (Edgar Allan Poe (1809 - 1849))}
    \end{flushright}
\end{epigrafe}
% ---

% -- resumo obrigatório
% ---
% RESUMOS
% ---

% resumo em português
\setlength{\absparsep}{18pt} % ajusta o espaçamento dos parágrafos do resumo
\begin{resumo}
FAZER POR ULTIMO DE ACORDO COM A MYLENA

 \textbf{Palavras-chaves}: APENAS PALAVRAS SEPARADAS POR PONTO
\end{resumo}

% resumo em inglês
\begin{resumo}[Abstract]
 \begin{otherlanguage*}{english}
   This is the english abstract.
   \vspace{\onelineskip}
 
   \noindent 
   \textbf{Key-words}: ONLY WORDS WITH DOT
 \end{otherlanguage*}
\end{resumo}
% ---
% inserir o sumario
% ---
\pdfbookmark[0]{\contentsname}{toc}
\tableofcontents*
\cleardoublepage
% ---


% ----------------------------------------------------------
% ELEMENTOS TEXTUAIS
% ----------------------------------------------------------
\textual


% ----------------------------------------------------------
% Introdução (exemplo de capítulo sem numeração, mas presente no Sumário)
% ----------------------------------------------------------
\chapter[Introdução]{Introdução}

\section{a}


\subsection{b}


\chapter{Sobre a InQuesti}
	A InQuesti é uma empresa de inteligência analítica, focada em inteligência de negócios e RPA (Robotic Process Automation, ou automação de processos com o uso de robôs) a empresa tem mais de 100 empresas atendidas, com mais de 500 projetos e com um time especializado. 

\section{Histórico da Empresa}
A InQuesti é uma empresa com 8 anos, foi fundada em 2012 por 3 sócios e atualmente, conta com apenas dois: Vinicius Siqueira e Luciano Kassen, tem projetos em vários tipos de empresa e consta com cerca de 20 funcionários, sendo que 5 deles são funcionários internos. 
Em dezembro de 2019, a InQuesti passou por uma reformulação, do seu planejamento estratégico, corpo de sócios e até a cultura organizacional (passando a ter um funcionário para cuidar disso). Essa reformulação impactou de forma crítica na sua forma de contratar seus colaboradores.

\section{Elementos públicos de planejamento estratégico}
O planejamento estratégico é uma das ferramentas de mais importância no desenvolvimento de uma empresa. Dentre vários pontos estratégicos a missão, visão e os valores serão abordados, por serem pontos com mais peso na hora de definir os objetivos e atingir as metas da empresa.

\subsection{Visão}
A visão da empresa são os objetivos a serem alcançados pela mesma em decorrencia de suas atividades. É extremamente necessário que sejamestabelecidos prazos (condizentes com a capacidade da empresa) para obter esses resultados, no mau estabelecimento desses objetivos a empresa pode arcar com uma certa desmotivação e perda de foco no andamento dos negócios. Sabendo disso, esta é a visão da InQuesti:

Ser a principal empresa de inteligência analítica do Brasil

\subsection{Missão}

A missão da empresa tem, em sua composição, tudo aquilo que a mesma pode oferecer para seus clientes (seja como prestação de serviço ou produto), na missção temos a descrição das caracteríscas chave do negócio, os principais pontos abordados pelo negócoio e as expectativas a serem resolvidas pela empresa.
A missão da InQuesti:

Somos uma empresa de tecnologia focada em inteligência de dados e temos como missão guiar o mundo através dos dados empoderando pessoas.

\subsection{Valores}
Os valores são todas as crenças que a empresa tem para cumprir suas atividades, nesse importante ponto do planejamento estratégico temos como destaque os valores pessoais dos empreendedores (baseados em suas experiência de vida), já que os valores são ideais a serem seguidos pelos colaboradores no decorrer de sua carreira na empresa.
Os valores da InQuesti:

1. Não escondemos, negamos ou negociamos aquilo que somos.

2. Fazemos o melhor em tudo o que nos comprometemos.

3. Cuidamos e zelamos por aquilo que conquistamos.

4. Evolução contínua é o nosso estilo de vida.

5. Mudamos o mundo, cumprimos nossa missão e deixamos nossa marca.
\chapter{Mecanismos de recrutamento, seleção e contratação da InQuesti}
% ---

% ---
\section{Etapas}


\section{Cronogramas}


\section{Análises}




% ---

\chapter{Análise dos mecanismos de recrutamento, seleção e contratação da InQuesti}
A análise dos mecanismos será apresentada em 3 partes: a primeira parte é uma reflexão sobre o mecanismo ideal para empresas do porte da InQuesti, a segunda 
é a comparação do atual mecanismo com o mecanismo ideal e a terceira parte é uma crítica e uma modelo proposto pelo autor, com mudanças
ou aprimoramentos. 

\section{Reflexão sobre o modelo ideal}
De acordo com a empresa Kenoby, um mecanismo correto e ideal de seleção passa pelas seguintes etapas:

1. Correta descrição do cargo em aberto: 


<RESUMIR>

Após receber e aprovar a requisição de pessoal, isto é, o documento formal que solicita o preenchimento de determinada vaga, é o momento de realizar a descrição do cargo em aberto. Assim, será possível ter maior alinhamento no R&S.


Para tanto, faça uma lista dos conhecimentos, habilidades e atitudes demandadas pelo cargo, bem como dos valores compartilhados na empresa. Esse será o “checklist” para contratar a pessoa certa, então é preciso fazer uma boa descrição do cargo.


2. Divulgação da vaga em aberto: <DESCRIÇÃO>;


<RESUMIR>

Em um terceiro momento, será necessário veicular o cargo em aberto pelos mais diversos canais de comunicação. Nesse ponto, o processo de recrutamento efetivamente começa. O intuito é atrair o maior número de talentos possível.


Entre os canais digitais para veiculação de vagas, é possível destacar:


página de carreira;
portais de vagas de emprego;
redes sociais;
plataformas de vídeo (como o YouTube).
No intuito de atrair um grande número de talentos, é comum que o profissional de RH opte por mais de uma plataforma de divulgação da vaga em aberto. Para tanto, é importante contar com um software de seleção, pois ele centraliza todo o comando.



3. Triagem dos candidatos mais promissores: <DESCRIÇÃO>;

4. Aplicação de testes e dinâmicas para seleção: <DESCRIÇÃO>;

5. Realização de entrevistas semiestruturadas: <DESCRIÇÃO>;

6. Seleção do candidato e contratação: <DESCRIÇÃO>.


\section{Comparação do atual mecanismo de recrutamento, seleção e contratação da InQuestide com o mecanismo ideal}


\section{Considerações, críticas e aprimoramentos}

% ---
% Conclusão (outro exemplo de capítulo sem numeração e presente no sumário)
% ---
\chapter*[Conclusão]{Conclusão}
\addcontentsline{toc}{chapter}{Conclusão}
% ---

lalalalalalalal


% ----------------------------------------------------------
% Finaliza a parte no bookmark do PDF
% para que se inicie o bookmark na raiz
% e adiciona espaço de parte no Sumário
% ----------------------------------------------------------
\phantompart

% ----------------------------------------------------------
% ELEMENTOS PÓS-TEXTUAIS
% ----------------------------------------------------------
\postextual
% ----------------------------------------------------------

% ----------------------------------------------------------
% Referências bibliográficas
% ----------------------------------------------------------
\bibliography{referencias,exemplos/abntex2-doc-abnt-6023}



%---------------------------------------------------------------------
% INDICE REMISSIVO - Quando necessário 
% As palavras indexadas devem ser definidas com \index{} no texto
%---------------------------------------------------------------------
\phantompart
\printindex


%---------------------------------------------------------------------

\end{document}