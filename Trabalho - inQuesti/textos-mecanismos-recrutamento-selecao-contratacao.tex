\chapter{Mecanismos de recrutamento, seleção e contratação da InQuesti}
A InQuesti, por ser uma empresa recente e com uma cultura organizacional reformulado, possui um único mecanismo de recrutamento para seus colaboradores, que consiste em várias etapas focadas na parte técnica de seus candidatos.

\section{Etapas}
O mecanismo de recrutamento da InQuesti é dividido em 5 etapas:
1º Triagem dos currículos dos candidatos: Aqui são feitas análises sobre a experiência profissional descrita no CV, para que a 2ª e 3ª etapas possam ser focadas no responsável correto.
2º Definição dos responsáveis pelo teste prático: Nessa etapa, o responsável da triagem escolhe um dos colaboradores para criar um teste que abranja a maioria dos conhecimentos técnicos descritos no CV dos candidatos.
3º Correção dos testes práticos: Nessa parte os testes são corrigidos pelos colaboradores responsáveis e após a correção, os mesmos repassam os melhores testes para o Vinicius (sócio e responsável pela parte de tecnologia da empresa)
4º Entrevista presencial: Vinicius então faz uma entrevista presencial com os candidatos.
5º Proposta salarial: Nessa parte final, a InQuesti entra em contato com uma proposta salarial CLT ou deixa em aberto uma contratação PJ.

\section{Cronogramas}
Cada etapa descrita tem um tempo máximo de realização:

1º: De 1 a 7 dias corridos.

2º: De 1 a 5 dias corridos.

3º: De 1 a 3 dias corridos.

4º: De 1 a 3 dias corridos.

5º: 1 dia corrido.

No mais tardar, a empresa leva cerca de 18 dias para finalizar seu processo seletivo.
