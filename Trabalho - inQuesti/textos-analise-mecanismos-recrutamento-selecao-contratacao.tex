\chapter{Análise dos mecanismos de recrutamento, seleção e contratação da InQuesti}
A análise dos mecanismos será apresentada em 3 partes: a primeira parte é uma reflexão sobre o mecanismo ideal para empresas do porte da InQuesti, a segunda 
é a comparação do atual mecanismo com o mecanismo ideal e a terceira parte é uma crítica e uma modelo proposto pelo autor, com mudanças
ou aprimoramentos. 

\section{Reflexão sobre o modelo ideal}
De acordo com a empresa Kenoby, um mecanismo correto e ideal de seleção passa pelas seguintes etapas:

1. Correta descrição do cargo em aberto: 


<RESUMIR>

Após receber e aprovar a requisição de pessoal, isto é, o documento formal que solicita o preenchimento de determinada vaga, é o momento de realizar a descrição do cargo em aberto. Assim, será possível ter maior alinhamento no R&S.


Para tanto, faça uma lista dos conhecimentos, habilidades e atitudes demandadas pelo cargo, bem como dos valores compartilhados na empresa. Esse será o “checklist” para contratar a pessoa certa, então é preciso fazer uma boa descrição do cargo.


2. Divulgação da vaga em aberto: <DESCRIÇÃO>;


<RESUMIR>

Em um terceiro momento, será necessário veicular o cargo em aberto pelos mais diversos canais de comunicação. Nesse ponto, o processo de recrutamento efetivamente começa. O intuito é atrair o maior número de talentos possível.


Entre os canais digitais para veiculação de vagas, é possível destacar:


página de carreira;
portais de vagas de emprego;
redes sociais;
plataformas de vídeo (como o YouTube).
No intuito de atrair um grande número de talentos, é comum que o profissional de RH opte por mais de uma plataforma de divulgação da vaga em aberto. Para tanto, é importante contar com um software de seleção, pois ele centraliza todo o comando.



3. Triagem dos candidatos mais promissores: <DESCRIÇÃO>;

4. Aplicação de testes e dinâmicas para seleção: <DESCRIÇÃO>;

5. Realização de entrevistas semiestruturadas: <DESCRIÇÃO>;

6. Seleção do candidato e contratação: <DESCRIÇÃO>.


\section{Comparação do atual mecanismo de recrutamento, seleção e contratação da InQuestide com o mecanismo ideal}


\section{Considerações, críticas e aprimoramentos}
