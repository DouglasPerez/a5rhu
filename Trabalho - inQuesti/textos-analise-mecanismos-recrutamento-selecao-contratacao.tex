\chapter{Mecanismos de recrutamento, seleção e contratação: apresentação de um modelo alternativo}
A análise dos mecanismos será apresentada em 3 partes: a primeira parte é uma reflexão sobre o mecanismo ideal para empresas do porte da InQuesti, a segunda 
é a comparação do atual mecanismo com o mecanismo ideal e a terceira parte 
é uma crítica e um modelo proposto pelo autor, com mudanças
ou aprimoramentos. 

\section{Apresentação de um modelo alternativo}
De acordo com a empresa Kenoby, um mecanismo correto e ideal de seleção passa pelas seguintes etapas:

1. Correta descrição do cargo em aberto: 
Nesta primeira etapa a vaga deve se ter toda a descrição da vaga, os conhecimentos, habilidades e atitudes demandadas para o cargo, também é uma boa prática anexar os valores compartilhados na empresa. Este documento, será a base para a contratação do profissional certo;

2. Divulgação da vaga em aberto: 
Agora, com os detalhes da vaga já decididos, é necessário divulgá-la nas redes sociais da empresa, canais de comunicação, plataformas de vídeo etc. O objetivo dessa divulgação é atrair o maior número de talentos possível;

3. Triagem dos candidatos mais promissores: 
Aqui temos o processo de triagem dos candidatos, faz-se uma classificação dos candidatos que possuem os pré-requisitos listados na primeira etapa, e eles seguem com o processo seletivo;

4. Aplicação de testes e dinâmicas para seleção:
Após a triagem dos candidatos, o próximo passo é a aplicação de testes, para que seja possível identificar os candidatos que tem as competências necessárias para a vaga. Existe uma infinidade de testes, mas a escolha do melhor (ou melhores) temdem a variar de acordo com o cargo;

5. Realização de entrevistas semiestruturadas:
A entrevista é um dos passos mais importantes no mecanismo de seleção, o cadidato fica cara-a-cara com o selecionador e/ou membros do time técnico para tirar conversar, tirar dúvidas e decidir sobre a contratação. Existem 3 tipos de entrevista, a estruturada (com perguntas previamente elaboradas), a não estruturada (em que não existe roteiro de perguntas) e semiestruturada (uma mistura entre as duas anteriores). O mais indicado atualmente é a entrevista semiestruturada, com um certo roteiro mas que permite uma certa flexibilidade para adaptação do selecionador;


6. Seleção do candidato e contratação:
Se o candidato cumprir todas as exigências necessárias pelo superior imediato do cargo em aberto, nesta etapa temos a contratação do colaborador, a negociação de valores e benefícios entre outros.

\subsection{Cronograma}
Cada etapa descrita tem um tempo estimado de realização:

1º: De 1 a 2 dias corridos.

2º: De 1 a 3 dias corridos.

3º: De 1 dia corridos.

4º: De 1 a 3 dias corridos.

5º: 1 dia corrido.

6º: 1 dia corrido.

No mais tardar, a o modelo prevê cerca de 11 dias para finalizar o processo seletivo.
