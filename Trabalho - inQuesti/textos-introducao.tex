
% ----------------------------------------------------------
% Introdução (exemplo de capítulo sem numeração, mas presente no Sumário)
% ----------------------------------------------------------
\chapter[Introdução]{Introdução}
O maior desafio no cenário profissional não é apenas se adaptar aos requisitos do mercado de contratação, mas também expor essa preparação às organizações por meio de processos seletivos.
De acordo com Gontijo (2005, p. 7), esses processos possuem duas exigências: rapidez de preencher a vaga e capacidade completa dos candidatos. Para a autora o desafio é:
traduzir as expectativas do cliente em um perfil passível de ser avaliado, mensurado, descrito em competências e que sinalize em direção à assertividade nas escolhas.

Para obter eficácia no processo seletivo é essencial que o perfil de competências a ser identificado esteja bem ajustado à demanda do detentor da vaga.
Morgado (2013, p. 89): 
As técnicas utilizadas na seleção de pessoas podem ser classificadas em cinco grupos: entrevista, seleção, prova de conhecimento ou capacidade, testes psicológicos, testes de personalidade e técnicas de simulação (apud, CHIAVENATO, 2009).

Na atual conjuntura, a globalização traz como consequência um ambiente mais competitivo nas organizações, sendo assim, atualmente é exigido muito mais dos candidatos do que experiência profissional e conhecimento acadêmico. Para Prado (2010, pg. 176): Na medida em que novas oportunidades surgem, ocorre uma dificuldade de adequação, tanto dos indivíduos, moldados pela antiga realidade, quanto das organizações movida pela nova economia.
Seguindo essa lógica, para selecionar profissionais com mais eficiência , as empresas devem promover uma forma de seleção também mais eficiente, reestruturando os métodos para o desenvolvimento pessoal e profissional e deixando de lado as técnicas tradicionais que visam pré determinar um candidato ideal.
Diante do conflito da adaptação mútua entre candidato e empresa frente às novas necessidades profissionais, levanta-se o questionamento acerca da eficácia dos processos seletivos atuais. 

Nessa perspectiva, o presente trabalho tem como objetivo analisar qualitativamente e discutir os mecanismos de contratação empresarial, com enfoque na InQuesti (empresa de tecnologia analítica e de automação de processos) por meio de entrevista com um funcionário que faz parte do meio de cultura organizacional da mesma.
