\chapter{Análise dos mecanismos de recrutamento, seleção e contratação}
O modelo de seleção da InQuesti possui muitas semelhanças ao modelo proposto, mas também existem
 diferenças que vão ser apontadas adiante:

A primeira etapa da inQuesti é comparável, basicamente, as 3 primeiras etapas do modelo proposto, enxutas em uma etapa só, o que pode causar um aumento no cronograma, visto a quantidade de atividades realizadas em apenas uma etapa.

Outra diferença que vale a pena ser apontada: a InQuesti utiliza dos seus colaboradores para a confecção dos testes técnicos e correções, hoje em dia isso pode ser automatizado para não ter um gasto de tempo do colaborador em tarefas que não estão no escopo da sua contratação.

Também é importante destacar que a parte de testes práticos da InQuesti foca apenas na parte técnica 
do colaborador, sem nenhuma outra dinâmica que não seja a o gerenciamento de tempo e capacidades técnicas,
 as outras competências são observadas em entrevistas, não abrangir mais 
competências nesses testes pode acarretar em uma quantidade desnecessária de entrevistas.

Outro ponto interessante é que a dinâmica de entrevista da InQuesti é não estruturada, o sócio Vinicius realiza as entrevistas presencialmente e/ou digitalmente.

\section{Considerações finais}
A InQuesti possui alguns problemas com seu modo de contratação, mas são problemas justificáveis visto que a empresa teve uma reformulação recente, e que ainda é uma empresa de pequeno porte (por conta disso não conta com profissionais capacitados para a área de Recursos Humanos e nem terceiriza este trabalho para que não tenha mais custos operacionais).

Porém, mesmo com essas justificativas, após o levantamento dos dados e comparação com um modelo alternativo, pode-se afirmar que, existem muitas melhorias que podem ser feitas sem necessitar mais investimento, comparado ao modelo alternativo a InQuesti tem uma demora de 7 dias para cumprimento das etapas de seleção, e isso pode ser um grande diferencial para que futuros colaboradores não optem pela escolha da empresa, visto que muitos fazem mais de um processo seletivo ao mesmo tempo, e o primeiro a dar a resposta normalmente vence essa corrida pela contratação.