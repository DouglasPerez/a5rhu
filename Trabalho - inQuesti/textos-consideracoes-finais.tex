\chapter{Análise dos mecanismos de recrutamento, seleção e contratação}
O modelo de seleção da InQuesti por mais semelhante do modelo proposto, possui as seguintes divergências:
A primeira etapa da inQuesti agrupa as 3 primeiras do modelo proposto, o que pode causar um aumento no cronograma, visto a quantidade de atividades realizadas em apenas uma etapa.

Outra diferença que vale a pena ser apontada é a utilização dos colaboradores por parte da empresa para confecção dos testes técnicos e correções. No momento atual, isso pode ser automatizado para que o colaborador não tenha que focar em tarefas que não estão no escopo de sua contratação. 

Também é importante destacar que os testes práticos da InQuesti focam apenas no conhecimento técnico do colaborador, sem nenhuma outra dinâmica que não seja a do gerenciamento de tempo e capacidades técnicas. Não abranger as demais competências nesses testes pode acarretar em uma quantidade desnecessária de entrevistas.

Por fim, a dinâmica de entrevista da InQuesti é não estruturada, o sócio Vinicius realiza as entrevistas presencialmente e/ou digitalmente.

De frente ao levantamento de dados, pode-se afirmar que, existem muitas melhorias que podem ser realizadas, uma vez que, a empresa mostra ser comprometida com reestruturações, como citado anteriormente.
Um dos motivos das divergências apontadas pode ser justificado pela empresa não possuir área de Recursos Humanos e nem terceirizar este trabalho para que não tenha mais custos operacionais.
Comparado ao modelo alternativo a InQuesti possui uma desvantagem de tempo no cumprimento das etapas de seleção, e isso pode ser um grande diferencial para que futuros colaboradores não optem pela escolha da empresa, visto que muitos fazem mais de um processo seletivo ao mesmo tempo, e o primeiro a dar a resposta normalmente vence essa corrida pela contratação.

\section{Considerações finais}
A seleção de profissionais é uma área de constante evolução pela necessidade de estar alinhada com as demandas da globalização.
Diante dos argumentos propostos, fica evidente não só os impasses da empresa frente ao modelo proposto mas também suas vantagens, visto que, mesmo sendo de pequeno porte possui interesse em adequar seus métodos de seleção, citado acima como a reestruturação recente e satisfação de seus funcionários e clientes.
Para garantir o sucesso nas práticas de seleções, é imprescindível que as empresas tenham comprometimento com a qualidade dos seus serviços para que desenvolvam a motivação de melhorar seus métodos de seleção de funcionários.
Por fim, é importante salientar a limitação desta pesquisa pela relação do autor e entrevistado com a empresa analisada, dessa forma, é imprudente a generalização de seus resultados e é fortemente recomendado uma análise externa e mais abrangente.
