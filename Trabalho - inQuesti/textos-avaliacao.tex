\chapter{Avaliação de funcionários da InQuesti}
A InQuesti não possui um método de avaliação dos funcionários, apenas apresentações do andamento dos projetos atuais e novas
técnicas empregadas nos mesmos (que não servem como mecanismos de avaliação de desempenho).

\section{Mecanismo de avaliação de funcionários: apresentação de um modelo alternativo}
É um processo feito pelas empresas para mensurar o desempenho e comportamentos dos funcionários a partir de critérios pré-estabelecidos.

\subsection{Avaliação de desempenho: Avaliação 360º}
Método que visa se basear nas opiniões de quem está trabalhando no projeto do colaborador, o feedback é feito de forma global.
O processo pode ser feito com formulários/questionários ou entrevistas, e ao final do processo deve-se ter um conjunto de dados e informações 
necessários para que tenha uma análise de desempenho satisfatória dos colaboradores.

\section{Análise dos mecanismos de desempenho da InQuesti}
Como dito anteriormente, a InQuesti não possui um modelo de mecanismo de desempenho, portanto este capítulo não analisará os mecanismos e sim 
fará a proposta do modelo apresentado anteriormente.
A avaliação de desempenho é importante para a empresa para não só quantificar o desempenho dos funcinários, mas também para 
fazer o estudo de custo por colaborador, verificar possíveis promoções e aumentos, aumentar a moral dos colaboradores.
Com isso podemos concluir que, a InQuesti se beneficiaria muito mais com esse modelo (simples porém eficaz) e que agregaria muito valor no seu time de colaboradores.